\documentclass[11pt,a4paper,oneside]{article}\usepackage[]{graphicx}\usepackage[]{color}
%% maxwidth is the original width if it is less than linewidth
%% otherwise use linewidth (to make sure the graphics do not exceed the margin)
\makeatletter
\def\maxwidth{ %
  \ifdim\Gin@nat@width>\linewidth
    \linewidth
  \else
    \Gin@nat@width
  \fi
}
\makeatother

\definecolor{fgcolor}{rgb}{0.345, 0.345, 0.345}
\newcommand{\hlnum}[1]{\textcolor[rgb]{0.686,0.059,0.569}{#1}}%
\newcommand{\hlstr}[1]{\textcolor[rgb]{0.192,0.494,0.8}{#1}}%
\newcommand{\hlcom}[1]{\textcolor[rgb]{0.678,0.584,0.686}{\textit{#1}}}%
\newcommand{\hlopt}[1]{\textcolor[rgb]{0,0,0}{#1}}%
\newcommand{\hlstd}[1]{\textcolor[rgb]{0.345,0.345,0.345}{#1}}%
\newcommand{\hlkwa}[1]{\textcolor[rgb]{0.161,0.373,0.58}{\textbf{#1}}}%
\newcommand{\hlkwb}[1]{\textcolor[rgb]{0.69,0.353,0.396}{#1}}%
\newcommand{\hlkwc}[1]{\textcolor[rgb]{0.333,0.667,0.333}{#1}}%
\newcommand{\hlkwd}[1]{\textcolor[rgb]{0.737,0.353,0.396}{\textbf{#1}}}%

\usepackage{framed}
\makeatletter
\newenvironment{kframe}{%
 \def\at@end@of@kframe{}%
 \ifinner\ifhmode%
  \def\at@end@of@kframe{\end{minipage}}%
  \begin{minipage}{\columnwidth}%
 \fi\fi%
 \def\FrameCommand##1{\hskip\@totalleftmargin \hskip-\fboxsep
 \colorbox{shadecolor}{##1}\hskip-\fboxsep
     % There is no \\@totalrightmargin, so:
     \hskip-\linewidth \hskip-\@totalleftmargin \hskip\columnwidth}%
 \MakeFramed {\advance\hsize-\width
   \@totalleftmargin\z@ \linewidth\hsize
   \@setminipage}}%
 {\par\unskip\endMakeFramed%
 \at@end@of@kframe}
\makeatother

\definecolor{shadecolor}{rgb}{.97, .97, .97}
\definecolor{messagecolor}{rgb}{0, 0, 0}
\definecolor{warningcolor}{rgb}{1, 0, 1}
\definecolor{errorcolor}{rgb}{1, 0, 0}
\newenvironment{knitrout}{}{} % an empty environment to be redefined in TeX

\usepackage{alltt}
\usepackage{amsmath,amsthm,amsfonts,amssymb}
\usepackage{pst-eucl,pstricks,pstricks-add}
%\usepackage[utf8]{inputenc}
%\usepackage[latin1]{inputenc}
\usepackage[spanish,activeacute]{babel}
\usepackage[a4paper,margin=2.5cm]{geometry}
\usepackage{times}
\usepackage[T1]{fontenc}
\usepackage{titlesec}
\usepackage{color}
\usepackage{url}
\usepackage{float}
\usepackage{cite}
\usepackage{graphicx}
\usepackage{multicol}
\usepackage{float}
\usepackage{lmodern}
\parindent=0mm
\IfFileExists{upquote.sty}{\usepackage{upquote}}{}
\begin{document}

\thispagestyle{empty}
{\sf
{\Large \scshape Escuela Polit\'{e}cnica Nacional} \hfill {\scshape 5 de Enero 2016}\\[3mm] 
{\scshape C\'{a}lculo en una variable \hfill Prueba $\#8$}\\[7mm]
{\scshape Nombre:} \rule{0.6\textwidth}{0.5pt}\qquad {\scshape Nro. lista:} \rule{0.1\textwidth}{0.5pt}\\
}




\begin{enumerate}
      \item Evalúe la integral indefinida dada.
      \begin{enumerate}
            \item {\scshape (1 Pto.)} $\displaystyle \int \frac{t^3-9t-6}{(3t)^4}\ dt$\\[35mm]
      \end{enumerate}
      \item Evalúe la integral indefinida dada usando una sustitución idónea.
      \begin{enumerate}
            \item {\scshape (1.5 Ptos.)} $\displaystyle \int \frac{1}{x(\ln x)^3}\ dx$\\[35mm]
            %\item {\scshape (1 Pto.)} $\displaystyle \int \cos^2 4x\ dx$
      \end{enumerate}
      \item {\scshape (3 Ptos.)} Aproxime el área A bajo la gráfica de $f(x)=x^3-x^2+2$ por medio de la suma de áreas de rectángulos sobre el intervalo $[1,3]$.
      
      \newpage
      \item Use el teorema fundamental del cálculo para evaluar la integral definida dada
      \begin{enumerate}
            %\item {\scshape (1 Pto.)} $\displaystyle \int_{-1/2}^{3/2} (x-\cos \pi x)\ dx$
            \item {\scshape (1.5 Ptos.)} $\displaystyle \int_{-2}^{2} \frac{u^3+u}{(u^4+2u^2-7)^5}\ du$\\[35mm]
      \end{enumerate}
      
      \item Use integración por partes para evaluar la integral dada
      \begin{enumerate}
            \item {\scshape (1 Pto.)} $\displaystyle \int \ln x^6\ dx$\\[38mm]
            \item {\scshape (2 Ptos.)} $\displaystyle \int \cos^4 x\ dx$\\[38mm]
      \end{enumerate}
      
      \item Evalúe la integral indefinida dada por medio de una sustitución trigonométrica
      \begin{enumerate}
            \item {\scshape (2 Ptos.)} $\displaystyle \int \frac{x-1}{(12-4x-x^2)^{3/2}}\ dx$
            %\item 
      \end{enumerate}
\end{enumerate}

\end{document}
